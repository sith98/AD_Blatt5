\documentclass[a4paper]{article}

\usepackage[ngerman]{babel}
\usepackage[T1]{fontenc}
\usepackage[utf8]{inputenc}
\usepackage[paper=a4paper,left=25mm,right=25mm,top=25mm,bottom=25mm]{geometry}
\usepackage{amsmath}
\usepackage{amssymb}
\usepackage{amsthm}

\newcommand{\BigO}{\operatorname{O}}
\newcommand{\T}{\operatorname{T}}
\newcommand{\ceil}[1]{\left\lceil{}{#1}\right\rceil}
\newcommand{\floor}[1]{\left\lfloor{}{#1}\right\rfloor}

\title{AD Übungsblatt 5}
\author{Simon Thelen}

\begin{document}
    \maketitle

    \section*{Aufgabe 2}
    \label{sec:aufgabe2}

    \subsection*{1}
    \label{subsec:aufgabe2_1}
    z.z.:\  Ein Heap mit $n$ Elementen hat Höhe $\lfloor\log{n}\rfloor$
    \begin{proof}
        Beweis durch Induktion

        \begin{description}
            \item[IA] $\lfloor\log{1}\rfloor = 0$
            \item[IV] Ein Heap mit $n$ Elementen hat Höhe $\lfloor\log{n}\rfloor$
            \item[IS] $n \longmapsto n + 1$
            \begin{description}
                \item[Fall 1] $n + 1 = 2^i, i \in \mathbb{N}$
                \begin{gather*}
                    \floor{\log{n + 1}} = \floor{\log{2^i}} = \log{2^i} = i = \floor{\log{n}} + 1
                \end{gather*}
                Stimmt, da bei einem vollen Heap gilt: $n = 2^i - 1$, was bedeutet, dass ein Baum mit $n + 1$ eins höher ist.
                \item[Fall 2] $n + 1 = 2^i - k, k \in \mathbb{N}, 1 < k < 2^{i - 1}$
                \begin{gather*}
                    n + 1 > 2^i - 2^{i - 1} = 2^{i - 1}
                    \Rightarrow \log{(n + 1)} > \log{2^{i - 1}} = i - 1
                    \Rightarrow \floor{\log{(n + 1)}} = \floor{\log{n}}
                \end{gather*}
                Stimmt, da der Heap bei $n$ noch nicht voll war. Das heißt die Höhe bleibt bei $n + 1$ gleich.
            \end{description}
        \end{description}
    \end{proof}

    \subsection*{2}
    \label{subsec:aufgabe2_2}
    z.z.:\  Ein Heap mit $n$ Elementen hat höchstens $\lceil\frac{n}{2^{h+1}}\rceil$ viele Knoten der Höhe $h$.

    \begin{proof} Beweis durch Induktion

        \begin{description}
            \item[IA] $n = 1$
            \begin{gather*}
                0 < \frac{1}{2^{k + 1}} < 1, \forall k \Rightarrow \ceil{\frac{1}{2^{k + 1}}} = 1
            \end{gather*}
            \item[IV] Ein Heap mit $n$ Elementen hat höchstens $\ceil{\frac{n}{2^{h+1}}}$ viele Knoten der Höhe $h$.
            \item[IS] $n \longmapsto n + 1$
            \begin{description}
                \item[IA] $k = 0$
                \begin{gather*}
                    \ceil{\frac{n + 1}{2}}
                \end{gather*}
                Stimmt, da sich die
            \end{description}
        \end{description}
    \end{proof}

    \subsection*{3}
    \label{subsec:aufgabe2_3}
    z.z.:~$\sum_{k=0}^{\infty}{k x^k} = \frac{x}{(1 - x)^2}$

    \begin{proof}
        \begin{align*}
            \sum_{k=0}^{\infty}{x^k} &= \frac{1}{1 - x} \\
            \left( \sum_{k=0}^{\infty}{x^k} \right)' &= \left( \frac{1}{1 - x} \right)' \\
            \sum_{k=0}^{\infty}{k x^{k-1}} &= \frac{1}{(1 - x)^2} \quad | \cdot x \\
            \sum_{k=0}^{\infty}{k x^x} &= \frac{x}{(1 - x)^2}
        \end{align*}
    \end{proof}

    \subsection*{Vertauschbarkeit von Heapify bei BuildHeap}
    Die Heapify-Aufrufe können nicht einfach vertauscht werden.
    Ein Gegenbeispiel wäre das Array $[1, 2, 3, 4]$.
    Wird hier Heapify wie im BuildHeap zuerst auf das zweite und dann auf das erste Element aufgerufen, ergibt sich der Heap $[4, 2, 3, 1]$.
    Dreht man die Reihenfolge jedoch um, lautet das Ergebnis $[3, 4, 1, 2]$, was kein korrekter Heap ist.

    \section*{Aufgabe 3}
    \label{sec:aufgabe3}
    \subsection*{Variante 1}
    \label{subsec:variante1}
    \begin{proof}
        Beweis durch Induktion

        \begin{description}
            \item[IA] für $n = 2$: $M \cdot N = O$ per Definition (normale Matrixmultipliktion für $\mathbb{R}^{2 \times 2}$)
            \item[IV] Berechnung der Multiplikation ist korrekt für $n$
            \item[IS] $n \longmapsto 2n$

            \setlength{\jot}{10pt}
            \begin{gather*}
                \begin{pmatrix}
                    M_{11} & M_{12} \\
                    M_{21} & M_{22} \\
                \end{pmatrix} \cdot
                \begin{pmatrix}
                    N_{11} & N_{12} \\
                    N_{21} & N_{22} \\
                \end{pmatrix} = \\
                \begin{pmatrix}
                    \begin{pmatrix}
                        M_{11,11} & M_{11,12} \\
                        M_{11,21} & M_{11,22}
                    \end{pmatrix} & \dots \\
                    \vdots & \ddots
                \end{pmatrix} \cdot
                \begin{pmatrix}
                    \begin{pmatrix}
                        N_{11,11} & N_{11,12} \\
                        N_{11,21} & N_{11,22}
                    \end{pmatrix} & \dots \\
                    \vdots & \ddots
                \end{pmatrix} = \\
                \begin{pmatrix}
                    \begin{pmatrix}
                        M_{11,11} & M_{11,12} \\
                        M_{11,21} & M_{11,22}
                    \end{pmatrix} \cdot
                    \begin{pmatrix}
                        N_{11,11} & N_{11,12} \\
                        N_{11,21} & N_{11,22}
                    \end{pmatrix} +
                    \begin{pmatrix}
                        M_{12,11} & M_{12,12} \\
                        M_{12,21} & M_{12,22}
                    \end{pmatrix} \cdot
                    \begin{pmatrix}
                        N_{21,11} & N_{21,12} \\
                        N_{21,21} & N_{21,22}
                    \end{pmatrix} & \dots \\
                    \vdots & \ddots
                \end{pmatrix} = \\
                \begin{pmatrix}
                    \left(\begin{smallmatrix}
                              M_{11,11} \cdot N_{11,11} + M_{11,12} \cdot N_{11,21} + M_{12,11} \cdot N_{21,11} + M_{12,12} \cdot N_{21,21} &
                              M_{11,11} \cdot N_{11,12} + M_{11,12} \cdot N_{11,22} + M_{12,11} \cdot N_{21,12} + M_{12,12} \cdot N_{21,22} \\
                              M_{11,21} \cdot N_{11,11} + M_{11,22} \cdot N_{11,21} + M_{12,21} \cdot N_{21,11} + M_{12,22} \cdot N_{21,21} &
                              M_{11,21} \cdot N_{11,12} + M_{11,22} \cdot N_{11,22} + M_{12,21} \cdot N_{21,12} + M_{12,22} \cdot N_{21,22}
                    \end{smallmatrix}\right) & \dots \\
                    \vdots & \ddots
                \end{pmatrix} = \\
                = \begin{pmatrix}
                      O_{11,11} & O_{11,12} & \dots \\
                      O_{11,21} & O_{11,22} & \dots \\
                      \vdots & \vdots & \ddots
                \end{pmatrix}
            \end{gather*}
            Das gleiche gilt analog auch für $O_{12}$, $O_{21}$ und $O_{22}$.
        \end{description}
    \end{proof}

    \subsubsection*{Laufzeitverhalten}
    \begin{align*}
        \T(n) =
        \begin{cases}
            1 & n = 2 \\
            8\T(n / 2) + \Theta(n^2) & n > 2
        \end{cases}
    \end{align*}
    Da $f(n) = \Theta(n^2) = \BigO(n^{\log_2(8)})$, gilt nach Fall 1 der Mastermethode $\T(n) = \Theta(n^{\log_2(8)}) = \Theta(n^3)$, was keine Verbesserung zur {\glqq}Standardmethode{\grqq} darstellt.

    \subsection*{Variante 2}
    \label{subsec:variante2}
    \begin{proof}
        Es gilt
        \begin{multline*}
            O_{11} = (M_{11}+M_{22})\cdot(N_{11}+N_{22}) + M_{22}\cdot(N_{21}-N_{11}) \\
            - (M_{11} + M_{12}) \cdot N_{22} + (M_{12}-M_{22})\cdot(N_{21}+N_{22}) \\
            = M_{11}\cdot N_{11} + M_{11}\cdot N_{22} + M_{22}\cdot N_{11} + M_{22}\cdot N_{22} + M_{22}\cdot N_{21} - M_{22}\cdot N_{11} \\
            - M_{11}\cdot N_{22} - M_{12}\cdot N_{22} + M_{12}\cdot N_{21} + M_{12}\cdot N_{22} - M_{22}\cdot N_{21} - M_{22}\cdot N_{22} \\
            = M_{11} \cdot N_{11} + M_{12} \cdot N_{21}
            \text{,}
        \end{multline*}
        was dem gleichen Wert für $O_{11}$ wie bei Variante~1 entspricht. Analog gilt das gleiche für $O_{12}$, $O_{21}$ und $O_{22}$.
        Unter der Annahme, dass Variante~1 korrekt ist, muss damit auch Variante~2 korrekt sein.
    \end{proof}

    \subsubsection*{Laufzeitverhalten}
    \begin{align*}
        \T(n) =
        \begin{cases}
            1 & n = 2 \\
            7\T(n / 2) + \Theta(n^2) & n > 2
        \end{cases}
    \end{align*}
    Da $f(n) = \Theta(n^2) = \BigO(n^{\log_{2}{7}})$, gilt nach dem 1. Fall der Mastermethode $\T(n) = \Theta(n^{\log_2{7}}) \approx \Theta(n^{2{,}81})$, was asymtotisch gesehen eine Verbesserung der Laufzeit ist.


\end{document}